\documentclass{beamer}
\usepackage[utf8]{inputenc}
\usepackage{amsmath}
\usepackage{tcolorbox}
\usepackage{multicol}
\usepackage{multirow}
\usepackage{tikz}
\usepackage{array}
\usepackage{colortbl}
\usetikzlibrary{arrows.meta,positioning,fit,shapes.misc}
\usetikzlibrary{tikzmark,overlay-beamer-styles,bending}
\usepackage{hyperref}
\hypersetup{
    colorlinks=true,
    linkcolor=blue,
    filecolor=magenta,      
    urlcolor=cyan,
    pdftitle={Overleaf Example},
    pdfpagemode=FullScreen,
}



\usetheme{Madrid}
\usecolortheme{spruce}

%------------------------------------------------------------
%This block of code defines the information to appear in the
%Title page
\title[Tabular method] %optional
{Boolean Function Simplification}

\subtitle{Tabular Method}

\author[Fahad, Masum] % (optional)
{
    Prepared by\\
    Abdullah Al Fahad(1805033)\\
    and\\
    Hasan Masum(1805052)
}

% \institute[BUET] % (optional)
% {
%   \inst{1}%
%   Student of CSE\\
%   Bangladesh University of Engineering and Technology
%   \and
%   \inst{2}%
%   Student of CSE\\
%     Bangladesh University of Engineering and Technology
% }

\date[30/08/2022] % (optional)
{30 August 2022}


\logo{\hspace*{8cm}~%
\includegraphics[width=1cm]{buet.png}
}

%End of title page configuration block
%------------------------------------------------------------
\definecolor{LightBlue}{RGB}{140,186,252}
\definecolor{row1}{RGB}{229,255,204}
\definecolor{row2}{RGB}{204,255,255}
\definecolor{gray}{RGB}{128, 128, 128}
\newcommand{\chk}{$\color{green}\checkmark$}
\newcommand{\ccr}{$\cellcolor{red!25}$}
\newcommand{\ccb}{$\cellcolor{blue!25}$}
\newcommand{\ccg}{$\cellcolor{green!25}$}
\newcommand{\hs}[1]{\hspace{#1}}
\newcommand{\mcl}[1]{\multicolumn{1}{l|}{#1}}

\begin{document}

%The next statement creates the title page.
\frame{\titlepage}
\logo{}

%---------------------------------------------------------
%This block of code is for the table of contents after
%the title page
\begin{frame}
    \frametitle{Table of Contents}
    \tableofcontents
\end{frame}
%---------------------------------------------------------


\section{Intruduction}

%---------------------------------------------------------
%%%%%%%%%%%%%%%% Slide-1: Introduction %%%%%%%%%%%%%%%%%%%%%%%%%%
\begin{frame}
    \frametitle{Introduction}
    \begin{itemize}
        \item<1-> Also known as Quine-McCluskey Tabular Method.
        \item<2-> No limit on the number of input variables.
        \item<3-> Can be programmed and implemented in a computer.
        \item<4-> Based on the concept of prime implicants.
    \end{itemize}
\end{frame}

%---------------------------------------------------------
%%%%%%%%%%%%%%%% Slide-2: Terminology %%%%%%%%%%%%%%%%%%%%%%%%%%
\begin{frame}
    \frametitle{Terminology}
    \begin{footnotesize}
            \begin{alertblock}{Implicants}
        Implicant is a product/minterm in Sum of Products (SOP) form or sum/maxterm in Product of Sums (POS) form of a Boolean function. E.g., consider a boolean function, $F = AB + ABC + BC.$ Implicants are AB, ABC and BC.
    \end{alertblock} \pause
    \begin{alertblock}{Prime implicants}
        A prime implicant of a function is an implicant that cannot be covered by a more general, (more reduced with fewer literals) implicant.
    \end{alertblock} \pause
    \begin{alertblock}{Essential prime implicants}
        Essential prime implicants (aka core prime implicants) are prime implicants that cover an output of the function that no combination of other prime implicants is able to cover.
    \end{alertblock}
    \end{footnotesize}
    
\end{frame}

%---------------------------------------------------------
%%%%%%%%%%%%%%%% Slide-3: How it works %%%%%%%%%%%%%%%%%%%%%%%%%%
\begin{frame}
    \frametitle{How it works?}
    This tabular method is useful to get the prime implicants by repeatedly using the following Boolean identity.
    \begin{equation*}
        xy+xy' = x(y+y') = x.1 = x
    \end{equation*}
    \begin{itemize}
        \item Two major steps
        \begin{itemize}
            \item Identify prime imlicants(implicant tables)
            \item Identify essential prime implicants(cover tables).
        \end{itemize}
    \end{itemize}
\end{frame}


%---------------------------------------------------------
\section{Example}

%---------------------------------------------------------
%%%%%%%%%%%%%%%% Slide-4: Example %%%%%%%%%%%%%%%%%%%%%%%%%%
\begin{frame}
    \frametitle{Example}
    \begin{example}
        \begin{math}
            Y(A,B,C,D) = \sum_{m} (2,6,8,9,10,11,14,15)
        \end{math}
        \\
        \alert{we will minimize it using \textit{Tabuler Method}}
    \end{example}
\end{frame}

%---------------------------------------------------------
%%%%%%%%%%%%%%%% Slide-5: Implicant Table %%%%%%%%%%%%%%%%%%%%%%%%%%
\begin{frame}{Implication table }
    Step-1: Group the minterm according to the number of 1's \\
    \begin{table}[]
        \begin{tabular}{b{1cm}|m{1.5cm}|m{2cm}}
            \hline
            \rowcolor{LightBlue}          Group & Minterm  & binary \\ \hline
            \rowcolor{row1}  1 & $m_2$    & 0 0 1 0 \\  \cline{2-3}
            \rowcolor{row1}    & $m_8$    & 1 0 0 0 \onslide<2-> \\
            \hline
            \rowcolor{row2}  2 & $m_6$    & 0 1 1 0 \\ \cline{2-3}
            \rowcolor{row2}    & $m_9$    & 1 0 0 1  \\ \cline{2-3} 
            \rowcolor{row2}    & $m_{10}$    & 1 0 1 0 \onslide<3->  \\ \hline
            \rowcolor{row1}  3 & $m_{11}$   & 1 0 1 1               \\ \cline{2-3}
            \rowcolor{row1}    & $m_{14}$ & 1 1 1 0  \onslide<4-> \\ \hline
            \rowcolor{row2}  4 & $m_{15}$ & 1 1 1 1               \\
        \end{tabular}
    \end{table}
\end{frame}

%%%%%%%%%%%%%%% step-2 %%%%%%%%%%%%%%%%%
\begin{frame}{Implication table}
    Step-2: Merge minterms from adjacent groups to form a new implication table.\\
    \begin{columns}
        \column{0.5\textwidth} % start of the column 1
        \begin{table}[]
            \begin{tabular}{|l|l|l|l|l|l|}
                \hline
                Group              & Min Term                                                                      & A & B & C & D                                                                                    \\ \hline
                \multirow{2}{*}{1} & $m_2$ & \mcl{0}      & \mcl{0} & \mcl{1} & 0 \\ \cline{2-6}
                                    & $m_8$ & \mcl{1}      & \mcl{0}                              & \mcl{0} & 0 \\ \hline
                \multirow{3}{*}{2} & $m_6$ & \mcl{0}      & \mcl{1} & \mcl{1} & 0 \\ \cline{2-6}
                                    & $m_9$ & \mcl{1}      & \mcl{0}                              & \mcl{0} & 1 \\ \cline{2-6}
                                    & $m_{10}$ & \mcl{1}      & \mcl{0}                              & \mcl{1} & 0 \\ \hline
                \multirow{2}{*}{3} & \tikzmarknode[]{node_m11}{$m_{11}$}                                           & \mcl{1}      & \mcl{0}                              & \mcl{1} & 1 \\ \cline{2-6}
                                    & \tikzmarknode[]{node_m14}{$m_{14}$}                                           & \mcl{1}      & \mcl{1}                              & \mcl{1} & 0 \\ \hline
                4                  & \tikzmarknode[]{node_m15}{$m_{15}$}                                           & \mcl{1}      & \mcl{1}                              & \mcl{1} & 1 \\ \hline
            \end{tabular}
        \end{table}
        \column{0.5\textwidth} % start of the column 2
    \end{columns}
\end{frame}



%---------------------------------------------------------
\begin{frame}{Implication table continue}
Step-2: Merge minterms from adjacent groups to form a new implication table. \\
    \begin{columns} % total 2 column
        \column{0.5\textwidth} %%%%%% 1st column
            \begin{table}[]
                \begin{tabular}{|l|l|l|l|l|l|}
                    \hline
                    Group              & Minterm                                                                      & A & B & C & D                                                                                     \\ \hline
                    %%%%% group 1 %%%%%%
                    \multirow{2}{*}{1} & \tikzmarknode[]{node_m2}{$m_2$} \hs{.5cm} \onslide<1->{\chk  }     & \mcl{\only<2>{\ccr0}\onslide<1,3->{0}} & \mcl{\only<1>{\ccr0}\onslide<2->{0}} & \mcl{1} & 0 \\ \cline{2-6}
                                    & \tikzmarknode[]{node_m8}{$m_8$} \hs{.5cm} \onslide<3->{\chk  }     & \mcl{1}      & \mcl{0} & \mcl{\only<4>{\ccr0}\onslide<1-3,5->{0}} & \mcl{\only<3>{\ccr0}\onslide<1-2,4->{0}} \\ \hline
                    %%% group 2 %%%%%
                    \multirow{3}{*}{2} & \tikzmarknode[]{node_m6}{$m_6$} \hs{.5cm} \onslide<1->{\chk  }     & \mcl{0}      & \mcl{\only<1>{\ccr1}\onslide<2->{1}} & \mcl{1} & 0 \\ \cline{2-6}
                                    & \tikzmarknode[]{node_m9}{$m_9$}\hs{.6cm} \onslide<3->{\chk  }      & \mcl{1}      & \mcl{0}                              & \mcl{0} & \mcl{\only<3>{\ccr1}\onslide<1-2,4->{1}} \\ \cline{2-6}
                                    & \tikzmarknode[]{node_m10}{$m_{10}$} \hs{0.3cm} \onslide<2->{\chk } &  \mcl{\only<2>{\ccr1}\onslide<1,3->{1}} & \mcl{0}  &  \mcl{\only<4>{\ccr1}\onslide<1-3,5->{1}} & 0 \\ \hline
                    %%%% group 3 %%%%
                    \multirow{2}{*}{3} & \tikzmarknode[]{node_m11}{$m_{11}$} \hs{0.3cm} \onslide<5->{\chk } & \mcl{1}      & \mcl{0}                              & \mcl{1} & 1 \\ \cline{2-6}
                                    & \tikzmarknode[]{node_m14}{$m_{14}$} \hs{0.3cm} \onslide<5->{\chk } & \mcl{1}      & \mcl{1}                              & \mcl{1} & 0 \\ \hline
                    4                  & \tikzmarknode[]{node_m15}{$m_{15}$} \hs{0.3cm} \onslide<6->{\chk } & \mcl{1}      & \mcl{1}                              & \mcl{1} & 1 \\ \hline
                \end{tabular}
            \end{table}

        \begin{tikzpicture}[overlay,remember picture]
            % start group1
            \draw<1>[<->, red] (node_m2) to [out=0,in=0] (node_m6);
            \draw<2>[<->, red] (node_m2) to [out=0,in=0] (node_m10);
            \draw<3>[<->, red] (node_m8) to [out=0,in=0] (node_m9);
            \draw<4>[<->, red] (node_m8) to [out=0,in=0] (node_m10);
            %end group one
            % start group2
            \draw<5>[<->, red] (node_m6) to [out=0,in=0] (node_m14);
            \draw<5>[<->, blue] (node_m9) to [out=0,in=0] (node_m11);
            \draw<5>[<->, cyan] (node_m10) to [out=0,in=0] (node_m11);
            \draw<5>[<->, orange] (node_m10) to [out=0,in=0] (node_m14);
            % end group2
            % start group3
            \draw<6>[<->, red] (node_m11) to [out=0,in=0] (node_m15);
            \draw<6>[<->, blue] (node_m14) to [out=0,in=0] (node_m15);
            % end group3
        \end{tikzpicture}
        
    
        \column{0.5\textwidth} %%%%% 2nd column
        % Please add the following required packages to your document preamble:
        % \usepackage{multirow}
        \begin{table}[]
            \begin{tabular}{l|l|llll}
                \hline
                \multicolumn{1}{|l|}{Group}         & Minterm        & \multicolumn{4}{l|}{Binary}     \\ \hline
                \rowcolor{row1} 1 \multirow{4}{*}{} & $m_2, m_6 $   & \mcl{0} & \mcl{-} & \mcl{1} & 0 \onslide<2-> \\ \cline{2-6}
                \rowcolor{row1}                     & $m_2, m_{10}$ & \mcl{-} & \mcl{0} & \mcl{1} & 0 \onslide<3-> \\ \cline{2-6}
                \rowcolor{row1}                     & $m_8, m_9$    & \mcl{1} & \mcl{0} & \mcl{0} & - \onslide<4-> \\ \cline{2-6}
                \rowcolor{row1}                     & $m_8, m_{10}$ & \mcl{1} & \mcl{0} & \mcl{-} & 0 \onslide<5-> \\ \hline
                \rowcolor{row2} 2 \multirow{4}{*}{} & $m_6, m_{14}$ & \mcl{-}  & \mcl{1}  & \mcl{1}  &    \onslide<5-> 0 \\ \cline{2-6}
                \rowcolor{row2}                     & $m_9, m_{11}$ & \mcl{1}   & \mcl{0}  & \mcl{-}  &    \onslide<5-> 0 \\ \cline{2-6}
                \rowcolor{row2}                     & $m_{10}, m_{11}$ & \mcl{1}       & \mcl{0}  & \mcl{1}  &\onslide<5->-\\ \cline{2-6}
                \rowcolor{row2}                     & $m_{10}, m_{14}$ & \mcl{1}       & \mcl{-}  & \mcl{1}  &  0  \onslide<6->\\ \hline
                \rowcolor{row1} 3 \multirow{2}{*}{} & $m_{11}, m_{15}$ & \mcl{1}       & \mcl{-}  & \mcl{1}  &  1 \onslide<6-> \\ \cline{2-6}
                \rowcolor{row1}                     & $m_{14}, m_{15}$ & \mcl{1}       & \mcl{1}  & \mcl{1}  &  -    \onslide<6-> \\ \hline
            \end{tabular}
        \end{table}
        
        \begin{table}[]
        \end{table}
    \end{columns}
\end{frame}

%%%%%%%%%%%%%%%%%%%%%%% 2nd pass %%%%%%%%%%%%%%%%%%%%%%%%%%%%%
\begin{frame}{Implication table continue}
Step-3: Repeat step 2 until no more merging is possible.
\begin{footnotesize}
     \begin{columns}
        \column{0.46\textwidth}
        \begin{table}[]
                \begin{tabular}{l|l|l|l|l|l|}
                \hline
                \multicolumn{1}{|l|}{Group}         & Minterm        & A & B & C & D      \\ \hline
                \rowcolor{row1} 1 \multirow{4}{*}{} & \tikzmarknode{m2_6}{$m_2,m_6$}   & \mcl{0} & \mcl{-} & \mcl{1} & 0  \\ \cline{2-6}
                \rowcolor{row1}                     & \tikzmarknode{m2_10}{$m_2, m_{10}$} & \mcl{-} & \mcl{0} & \mcl{1} & 0 \\ \cline{2-6}
                \rowcolor{row1}                     & \tikzmarknode{m8_9}{$m_8, m_9$}    & \mcl{1} & \mcl{0} & \mcl{0} & -  \\ \cline{2-6}
                \rowcolor{row1}                     & \tikzmarknode{m8_10}{$m_8, m_{10}$} & \mcl{1} & \mcl{0} & \mcl{-} & 0  \\ \hline
                \rowcolor{row2} 2 \multirow{4}{*}{} & \tikzmarknode{m6_14}{$m_6, m_{14}$} & \mcl{-}  & \mcl{1}  & \mcl{1}  &  0 \\ \cline{2-6}
                \rowcolor{row2}                     & \tikzmarknode{m9_11}{$m_9, m_{11}$} & \mcl{1}   & \mcl{0}  & \mcl{-}  &   0 \\ \cline{2-6}
                \rowcolor{row2}                     & \tikzmarknode{m10_11}{$m_{10}, m_{11}$} & \mcl{1}       & \mcl{0}  & \mcl{1}  & -\\ \cline{2-6}
                \rowcolor{row2}                     & \tikzmarknode{m10_14}{$m_{10}, m_{14}$} & \mcl{1}       & \mcl{-}  & \mcl{1}  &  0  \\ \hline
                \rowcolor{row1} 3 \multirow{2}{*}{} & \tikzmarknode{m11_15}{$m_{11}, m_{15}$} & \mcl{1}       & \mcl{-}  & \mcl{1}  &  1 \\ \cline{2-6}
                \rowcolor{row1}                     & \tikzmarknode{m14_15}{$m_{14}, m_{15}$} & \mcl{1}       & \mcl{1}  & \mcl{1}  &  - \\ \hline
            \end{tabular}
           % \end{tabular}
        \end{table}


        \begin{tikzpicture}[overlay,remember picture]
            % start group1
            \draw<1>[<->, red] (m2_6) to [out=0,in=0] (m10_14);
            \draw<1>[<->, blue] (m2_10) to [out=0,in=0] (m6_14);
            \draw<1>[<->, cyan] (m8_9) to [out=0,in=0] (m10_11);
            \draw<1>[<->, orange] (m8_10) to [out=0,in=0] (m9_11);
            %end group one
            % start group2
            \draw<2>[<->, red] (m10_11) to [out=0,in=0] (m14_15);
            \draw<2>[<->, blue] (m10_14) to [out=0,in=0] (m11_15);
            % end group2
        \end{tikzpicture}
        
        
        \column{0.6\textwidth}
        % Please add the following required packages to your document preamble:
        % \usepackage{multirow}
        \begin{table}[]
            \begin{tabular}{l|l|llll}
                \hline
                \multicolumn{1}{|l|}{Group}         & Minterm        & \multicolumn{4}{l|}{Binary}     \\ \hline
                \rowcolor{row1} 1 \multirow{4}{*}{} & $m_2,m_6,m_{10},m_{14}$   & \mcl{-} & \mcl{-} & \mcl{1} & 0  \\ \cline{2-6}
                \rowcolor{row1}                     & $m_2,m_{10},m_6,m_{14}$   & \mcl{-} & \mcl{-} & \mcl{1} & 0  \\ \cline{2-6}
                \rowcolor{row1}                     & $m_8,m_9,m_{10},m_{11}$ & \mcl{1} & \mcl{0} & \mcl{-} & - \\ \cline{2-6}
                \rowcolor{row1}                     & $m_8,m_{10},m_9,m_{11}$ & \mcl{1} & \mcl{0} & \mcl{-} & - \onslide<2-> \\ \hline
                \rowcolor{row2} 2 \multirow{4}{*}{} & $m_{10},m_{11},m_{14},m_{15}$ & \mcl{1}  & \mcl{-}  & \mcl{1}  &  - \\ \cline{2-6}
                \rowcolor{row2}                     & $m_{10},m_{14},m_{11},m_{15}$ & \mcl{1}  & \mcl{-}  & \mcl{1}  &  -  \\ \hline
            \end{tabular}
        \end{table}
    \end{columns}
\end{footnotesize}
   
\end{frame}

%%%%% 3rd pass %%%%
%%%%%%%%%%%%%%%%%%%%%%% 2nd pass %%%%%%%%%%%%%%%%%%%%%%%%%%%%%
\begin{frame}{Implication table continue}
\begin{footnotesize}
        \begin{table}[]
        \begin{tabular}{l|l|l|l|l|l|}
            \hline
            \multicolumn{1}{|l|}{Group}         & Minterm        & A & B & C & D      \\ \hline
            \rowcolor{row1} 1 \multirow{4}{*}{} & $m_2,m_6,m_{10},m_{14}$   & \mcl{-} & \mcl{-} & \mcl{1} & 0  \\ \cline{2-6}
            \rowcolor{row1}                     & $m_2,m_{10},m_6,m_{14}$   & \mcl{-} & \mcl{-} & \mcl{1} & 0  \\ \cline{2-6}
            \rowcolor{row1}                     & $m_8,m_9,m_{10},m_{11}$ & \mcl{1} & \mcl{0} & \mcl{-} & - \\ \cline{2-6}
            \rowcolor{row1}                     & $m_8,m_{10},m_9,m_{11}$ & \mcl{1} & \mcl{0} & \mcl{-} & -  \\ \hline
            \rowcolor{row2} 2 \multirow{4}{*}{} & $m_{10},m_{11},m_{14},m_{15}$ & \mcl{1}  & \mcl{-}  & \mcl{1}  &  - \\ \cline{2-6}
            \rowcolor{row2}                     & $m_{10},m_{14},m_{11},m_{15}$ & \mcl{1}  & \mcl{-}  & \mcl{1}  &  -  \\ \hline
        \end{tabular}
    \end{table}
    \pause
    The reduced table after removing the redundant rows is shown below:
    
    % Please add the following required packages to your document preamble:
    % \usepackage{multirow}
    \begin{table}[]
        \begin{tabular}{l|l|llll}
            \hline
            \multicolumn{1}{|l|}{Group}         & Minterm        & \multicolumn{4}{l|}{Binary}     \\ \hline
            \rowcolor{row1} 1 \multirow{4}{*}{} & $m_2,m_6,m_{10},m_{14}$   & \mcl{-} & \mcl{-} & \mcl{1} & 0  \\ \cline{2-6}
            \rowcolor{row1}                     & $m_8,m_9,m_{10},m_{11}$ & \mcl{1} & \mcl{0} & \mcl{-} & - \\ \hline
            \rowcolor{row2} 2 \multirow{4}{*}{} & $m_{10},m_{11},m_{14},m_{15}$ & \mcl{1}  & \mcl{-}  & \mcl{1}  &  - \\ \hline
        \end{tabular}
    \end{table}

\end{footnotesize}
\pause
\textcolor{red}{No more merging possible}  
\end{frame}

\begin{frame}{Implication table continue}
    \begin{table}[]
        \begin{tabular}{l|l|llll}
            \hline
            \multicolumn{1}{|l|}{Group}         & Minterm        & \multicolumn{4}{l|}{Binary}     \\ \hline
            \rowcolor{row1} 1 \multirow{4}{*}{} & $m_2,m_6,m_{10},m_{14}$   & \mcl{-} & \mcl{-} & \mcl{1} & 0  \\ \cline{2-6}
            \rowcolor{row1}                     & $m_8,m_9,m_{10},m_{11}$ & \mcl{1} & \mcl{0} & \mcl{-} & - \\ \hline
            \rowcolor{row2} 2 \multirow{4}{*}{} & $m_{10},m_{11},m_{14},m_{15}$ & \mcl{1}  & \mcl{-}  & \mcl{1}  &  - \\ \hline
        \end{tabular}
    \end{table}
    There are three rows in the above table. So, each row will give one prime implicant. Therefore, the prime implicants are $CD’$, $AB’$ \& $AC$.
\end{frame}

% \begin{frame}{Cover table}
%     \begin{table}[]
%         \begin{tabular}{|c|c|c|c|c|c|c|c|c|}
%             \hline
%             \begin{tabular}[c]{@{}c@{}}Min terms/ Prime\\  Imlicants\end{tabular} & 2 & 6 & 8 & 9 & 10 & 11 & 14 & 15 \\ \hline
%             CD'                                                                   & 1  & 1   &   &   & 1  &    & 1  &    \\ \hline
%             AB'                                                                   &   &   & 1 & 1 & 1  & 1  &    &    \\ \hline
%             AC                                                                    &   &   &   &   & 1  & 1  & 1  & 1  \\ \hline
%         \end{tabular}
%     \end{table}
% \end{frame}

\begin{frame}{Cover table}
    \begin{table}[]
        \begin{tabular}{|c|c|c|c|c|c|c|c|c|}
            \hline
            \begin{tabular}[c]{@{}c@{}}Min terms/ Prime\\  Imlicants\end{tabular} & \tikzmarknode{mn2}{2} & \tikzmarknode{mn6}{6} & \tikzmarknode{mn8}{8} & \tikzmarknode{mn9}{9} & \tikzmarknode{mn10}{10} & \tikzmarknode{mn11}{11} & \tikzmarknode{mn14}{14} & \tikzmarknode{mn15}{15} \\ \hline
            \tikzmarknode{cd_prime}{CD'}                                                                   & \onslide<1>{1}\only<2->{\ccr1}   & \onslide<1>{1}\only<2->{\ccr1}   &   &   & 1  &    & 1  & \tikzmarknode{cd_prime_end}{}   \\ \hline
            \tikzmarknode{ab_prime}{AB'}                                                                   &   &   & \onslide<1-2>{1}\only<3->{\ccb1} & \onslide<1-2>{1}\only<3->{\ccb1}  & 1  & 1  &    &  \tikzmarknode{ab_prime_end}{}  \\ \hline
            AC   & \tikzmarknode{mn2_e}{}  & \tikzmarknode{mn6_e}{}  & \tikzmarknode{mn8_e}{}  & \tikzmarknode{mn9_e}{}  & \tikzmarknode{mn10_e}{1}  & \tikzmarknode{mn11_e}{1}  & \tikzmarknode{mn14_e}{1}  & \onslide<1-3>{1}\only<4->{\ccg1} \tikzmarknode{mn15_e}{}  \\ \hline
        \end{tabular}
    \end{table}
    
    \begin{tikzpicture}[overlay,remember picture]
        \draw<2->[-, red] (cd_prime) to (cd_prime_end);
        \draw<3->[-, red] (ab_prime) to (ab_prime_end);
        \draw<2->[-, red] (mn2) to (mn2_e);
        \draw<2->[-, red] (mn6) to (mn6_e);
        \draw<2->[-, red] (mn10) to (mn10_e);
        \draw<2->[-, red] (mn14) to (mn14_e);
        
        \draw<3->[-, red] (mn8) to (mn8_e);
        \draw<3->[-, red] (mn9) to (mn9_e);
        \draw<3->[-, red] (mn11) to (mn11_e);
        %\draw<3->[-, red] (mn14) to (mn14_e);
    \end{tikzpicture}
    
    \setbeamercovered{transparent}
    \begin{itemize}
        \item<2-> The min terms 2 and 6 are covered only by one prime implicant CD'. So, it is an essential prime implicant.
        \item<3-> The min terms 8 and 9 are covered only by one prime implicant AB’. So, it is an essential prime implicant.
        \item<4-> The min terms 8 and 9 are covered only by one prime implicant AB’. So, it is an essential prime implicant.
    \end{itemize}
\end{frame}

\begin{frame}{Smplified Expression}
    We got three prime implicants and all the three are essential. Therefore, the simplified Boolean function is,
    \begin{equation*}
        Y(A,B,C,D) = CD' + AB' + AC
    \end{equation*}
\end{frame}

\section{Summary}
\begin{frame}{Summary of the method}
    \begin{itemize}
        \item Step-1: Group the minterm according to the number of 1’s. \pause
        \item Step-2: Compare and merge the min terms present in successive groups. \pause
        \item Step-3: Repeat step-2 with newly formed terms till we get all prime implicants. \pause
        \item Step-4: Formulate prime implicants table(cover table) and reduce it removing the row of each essential prime implicant and the columns corresponding to the min terms.
    \end{itemize}
\end{frame}{}

\begin{frame}{References} 
\begin{figure}
    \begin{itemize}
        \item \href{https://www.tutorialspoint.com/digital_circuits/digital_circuits_quine_mccluskey_tabular_method.htm}{Tutorials point}
        \item \href{https://en.wikipedia.org/wiki/Implicant}{https://en.wikipedia.org/wiki/Implicant}
    \end{itemize}
\end{figure}{}
\end{frame}{}


\end{document}